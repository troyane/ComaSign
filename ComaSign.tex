% ===
% Author: Nazar Gerasymchuk, 08.03.2013
% 	Nazar.Gerasymchuk [at] GMail
% 	http://neval8.wordpress.com/
% Feel free to improve this code, 
% or feed back to me.
% ===

\documentclass[a5paper, 14pt, oneside]{scrartcl}
\usepackage[T2A]{fontenc}				% внутрішнє кодування  TeX
\usepackage[utf8]{inputenc}				% кодова сторінка документа
\usepackage[english, ukrainian]{babel}	% локалізація і переноси
\usepackage{cmap}						% адекватний пошук в pdf
\usepackage{layouts}

% === ОПИС КОМАНДИ \ComaSign ===
% 0. Майже те що треба вже є готове:
%	\cb{!}
% 1. Описую змінну ComaLen -- ширина символу ","
%	\newlength{\ComaLen}
% 2. Для типової гарнітури -- змінній ComaLen покладаю значення ширини символу ","
%	\settowidth{\ComaLen}{,}
% 3. Друкую символ "!"
% 4. Засобами makebox досягаю друку коми зі зміщенням саме на необхідному місці
%	makebox шириною в ширину символу ",", в який поміщаю текст ",.." -- де
%	дві крапки -- займають місце, проте невилимі (\phantom)
\newlength{\ComaLen}
\newcommand{\ComaSign}{\settowidth{\ComaLen}{,}!\makebox[\ComaLen]{,\phantom{..}}}
% ===

\author{Nazar Gerasymchuk}

\begin{document}
% Друкую значення ComaLen в міліметрах за допомогою підключеного пакету layouts
%	\usepackage{layouts}
% Ширина символа <<,>> = \printinunitsof{mm}\prntlen{\ComaLen}

\today\vspace{2cm}

ComaSign у різних гарнітурах:
\begin{itemize}
	\item Типова: \ComaSign
	\item Курсив (\textit{italic}): \textit{\ComaSign}
	\item Напівжирний (\textbf{boldface}): \textbf{\ComaSign}
	\item Напівжирний курсив (\textbf{\textit{boldface italic}}): \textbf{\textit{\ComaSign}}
	\item Нахилений (\textsl{slanted}): \textsl{\ComaSign}
	\item Друкарська машинка (\texttt{typewriter}): \texttt{\ComaSign}
	\item Рублений (\textsf{sans serif}): \textsf{\ComaSign}
\end{itemize}
\end{document}